\documentclass{acm_proc_article-sp}

\begin{document}

\title{Voronoi Treemaps in D3}

\numberofauthors{2}
\author{
\alignauthor
Peter Henry
       \affaddr{University of Washington}\\
       \email{phenry@gmail.com}
\alignauthor
Paul Vines
       \affaddr{University of Washington}\\
       \email{paul.l.vines@gmail.com}
}
\date{21 March 2014}

\maketitle
\begin{abstract}
Blah blah blah
\end{abstract}


\section{Introduction}
Treemaps are a category of visualizations used for displaying
hierarchical data. While node-and-edge diagrams are often used for
visualizing hierachical structures, treemaps offer some significant
advantages. Primarily, treemaps are space-filling, and therefore allow
each node in a hierarchy to have more viewing area devoted to it than
in a node-and-edge diagram. This allows both larger hierarchies to be
visualized, as well as more detail to be shown on each node, such as
additional text, colors, or glyphs to show attributes of the
node. 

The majority of treemap layouts used are variants of rectangular
treemaps. These Have the advantage of being relatively fast to layout
and in cases of limited scale produce reasonably understandable
treemaps. However, there are three drawbacks to rectangular
treemaps. 

First, as hierarchies become deeper, the treemap cells can
become increasingly extreme in aspect ratio, resulting in narrow
rectangles more difficult to see than if their area was distributed in
a more square-like space. This problem is mostly mitigated by various
tweaks to the treemapping algorithm to try to keep the aspect ratio of
regions close to one. 

Second, the borders between different regions in the hiearchy can
become difficult to see. In particular, two cells
neighboring one another in the treemap but not siblings in the
hierarchy can appear to share a common edge delineating the same inner
node is their parent, when this is in fact not the case. 
Finally, rectangular treemap algorithms naturally only fill
rectangular regions which could be undesirable for aesthetic or
practical reasons.

Voronoi treemaps eliminate these problems. Firstly, Voronoi treemap
cells are arbitrary polygons but as will be discussed later, the
generation algorithm results in generally low aspect ratio
cells. Secondly, the fact that Voronoi treemap cells are arbitrary
polygons means edges between cells will fall at any angle, rather
than only vertical or horizontal, and so two neighboring cells will
generally never have a continuous-looking edge unless they are in fact
siblings in the hierarchy and thus share the edge of their parent
node's cell. Finally, Voronoi treemaps can be produced for any
arbitrary polygonal region, and so do not restrict the shape to be
filled by the treemap.

Multiple Voronoi treemap algorithms have been created in recent years
(CITE). However, none are available for use in a web framework.  Our
work has been to implement one of the fastest algorithms (CITE) for
use in the D3 web framework. Despite the optimizations employed by the
algorithm creators, generation of a Voronoi treemap is still a
computationally intensive task. Therefore, we have additionally
written the D3 module with features to try to allow Voronoi treemaps
to be used for web visualizations without causing a poor user
experience even on complex datasets.

The remainder of the paper is structured as follows: Section 2 has a
discussion of related work including a brief introduction to Weighted Voronoi
diagrams and a discussion of the algorithms created for Voronoi
treemaps. Section 3 describes the implementation of our work in D3 and
optimizations added for client-side web usability. Section 4 shows the
use of our framework on several datasets and an evaluation of the
computational burden of our system. Section 5 discusses the potential
applications of our system. Section 6 concludes with proposals of
future work to be done in this space.

\section{Related Work}

\subsection{Voronoi Diagrams}
Voronoi diagrams are a technique for dividing a region containing
sites into cells to satisfy the following condition: given a distance
function $d(p, q)$ where $p$ and $q$ are points, any point $p$ is
labeled as belonging to the site $q$ that results in the lowest
distance, $d(p,q)$. In this case to be labeled means to be inside a
bounding polygon formed for each site. In the case of a simple
euclidean distance function, $d(p,q) = \sqrt({dx}^2 + {dy}^2)$ this
results in a cell border being equidistant between the two closest
sites. 

For Voronoi treemaps two extensions are made to the basic Voronoi
diagram. First, sites are given initially random positions, a diagram
is generated, and then sites are moved to the centroidal positions in
their cell and then the diagram is re-generated. This is repeated
until a relatively stable set of positions is found (Lloyd). The effect of
this iterative process is to create lower aspect ratio cells. Second,
rather than using a standard euclidean distance function the
generation algorithm uses a weighted distance function, where each
site is assigned a weight that corresponds to generating a larger or
smaller cell. This allows the sizes of cells to be adjusted to reflect
the relative size or number of children of a specific node in the
hierarchy being displayed. 

After these extensions are made, the Voronoi treemap algorithm
proceeds to compute the Voronoi diagram for each level of the
hierarchy: it starts at the highest level, generates the Voronoi
diagram of the first level of nodes, and then recursively descends
into each cell and generates the Voronoi diagram for the children of
that node using the cell as the new bounding region. The computational
burden of this can be high; several different algorithms for computing
the Voronoi diagram have been developed and are briefly summarized
below.

\subsection{Previous Approaches}
Voronoi treemaps have been implemented previously (Balzer) using both
additively weighted and geometrically weighted Voronoi diagram
algorithms. This implementation used the same iterative algorithm for
creating centroidal Voronoi diagrams described above. To create the
weighted diagrams, however, it used a sampling algorithm wherein
points were sampled in the space and distances to nearest sites
calculated, to give an approximation of the correct weighted Voronoi
diagram. This results in an algorithm on the order of $O(n^2)$ where
$n$ is the number of sites. The benefit of this algorithm is that it
the sampling process is the bottleneck and is easily parallelized to
achieve linear speedups with additional CPU cores.

This algorithm implementation was improved by Sud et al. (CITE SUD) by
using GPU programming to significantly speedup computation by
parallelizing across graphics hardware. However, the algorithm
remained $O(n^2)$ for the number of sites. Further, this approach is
not feasible for web programming because consumer devices are not
commonly equipped with powerful graphics cards and do not all support
the use of the graphics card by a website (CITE).

The algorithm proposed by Nocaj \& Brandes (CITE NOCAJ) offers a
significant asymptotic improvement on these previous designs. Rather
than a sampling-based approach, this implementation uses the algorithm
for computing arbitrary-dimension Power Diagrams proposed by
Aurenhammer (CITE AURENHAMMER). In this approach the 2D points
representing sites are lifted into 3-dimensional dual space based on
their weights. The convex hull made by these 3D points is then
computed, and projected back down to 2D to produce the Voronoi
diagram. This method is on the order of $O(n \log n)$ and so can
provide a significant speedup for generating treemaps of larger
datasets. 

\section{Methods}
The core computational components of our implementation were adapted
from a Java implementation of the Nocaj \& Brandes algorithm (CITE
NOCAJ) using a lift into 3-dimensions followed by computation of
the convex hull and projection back into 2-dimensions to create the
Voronoi diagram. As with other implementations, we use Lloyd's
algorithm to iteratively adjust the site locations to be the centroids
of their cells and then adjust the weights of the sites to fit the
area of each cell to within an error threshold.


\section{Results}


\section{Discussion}


\section{Future Work}


\section{Acknowledgments}


\section{References}

\end{document}
