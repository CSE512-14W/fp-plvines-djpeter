\documentclass{acm_proc_article-sp}

\begin{document}

\title{Voronoi Treemaps in D3}

\numberofauthors{2}
\author{
\alignauthor
Peter Henry
       \affaddr{University of Washington}\\
       \email{phenry@gmail.com}
\alignauthor
Paul Vines
       \affaddr{University of Washington}\\
       \email{paul.l.vines@gmail.com}
}
\date{21 March 2014}

\maketitle
\begin{abstract}
Blah blah blah
\end{abstract}


\section{Introduction}
Treemaps are a category of visualizations used for displaying
hierarchical data. While node-and-edge diagrams are often used for
visualizing hierachical structures, treemaps offer some significant
advantages. Primarily, treemaps are space-filling, and therefore allow
each node in a hierarchy to have more viewing area devoted to it than
in a node-and-edge diagram. This allows both larger hierarchies to be
visualized, as well as more detail to be shown on each node, such as
additional text, colors, or glyphs to show attributes of the
node. 

The majority of treemap layouts used are variants of rectangular
treemaps. These Have the advantage of being relatively fast to layout
and in cases of limited scale produce reasonably understandable
treemaps. However, there are three drawbacks to rectangular
treemaps. 

First, as hierarchies become deeper, the treemap cells can
become increasingly extreme in aspect ratio, resulting in narrow
rectangles more difficult to see than if their area was distributed in
a more square-like space. This problem is mostly mitigated by various
tweaks to the treemapping algorithm to try to keep the aspect ratio of
regions close to one. 

Second, the borders between different regions in the hiearchy can
become difficult to see. In particular, two cells
neighboring one another in the treemap but not siblings in the
hierarchy can appear to share a common edge delineating the same inner
node is their parent, when this is in fact not the case. 
Finally, rectangular treemap algorithms naturally only fill
rectangular regions which could be undesirable for aesthetic or
practical reasons.

Voronoi Treemaps eliminate these problems. Firstly, Voronoi Treemap
cells are arbitrary polygons but as will be discussed later, the
generation algorithm results in generally low aspect ratio
cells. Secondly, the fact that Voronoi Treemap cells are arbitrary
polygons means edges between cells will fall at any angle, rather
than only vertical or horizontal, and so two neighboring cells will
generally never have a continuous-looking edge unless they are in fact
siblings in the hierarchy and thus share the edge of their parent
node's cell. Finally, Voronoi Treemaps can be produced for any
arbitrary polygonal region, and so do not restrict the shape to be
filled by the treemap.

Multiple Voronoi Treemap algorithms have been created in recent years
(CITE). However, none are available for use in a web framework.  Our
work has been to implement one of the fastest algorithms (CITE) for
use in the D3 web framework. Despite the optimizations employed by the
algorithm creators, generation of a Voronoi Treemap is still a
computationally intensive task. Therefore, we have additionally
written the D3 module with features to try to allow Voronoi Treemaps
to be used for web visualizations without causing a poor user
experience even on complex datasets.

The remainder of the paper is structured as follows: Section 2 has a
discussion of related work including a brief introduction to Weighted Voronoi
Diagrams and a discussion of the algorithms created for Voronoi
Treemaps. Section 3 describes the implementation of our work in D3 and
optimizations added for client-side web usability. Section 4 shows the
use of our framework on several datasets and an evaluation of the
computational burden of our system. Section 5 discusses the potential
applications of our system. Section 6 concludes with proposals of
future work to be done in this space.

\section{Related Work}


\section{Methods}


\section{Results}


\section{Discussion}


\section{Future Work}


\section{Acknowledgments}


\section{References}

\end{document}
